% Created 2025-12-07 Sun 15:28
% Intended LaTeX compiler: pdflatex
\documentclass[11pt]{article}
\usepackage[utf8]{inputenc}
\usepackage[T1]{fontenc}
\usepackage{graphicx}
\usepackage{longtable}
\usepackage{wrapfig}
\usepackage{rotating}
\usepackage[normalem]{ulem}
\usepackage{amsmath}
\usepackage{amssymb}
\usepackage{capt-of}
\usepackage{hyperref}
\author{Gabriel Rolland}
\date{\today}
\title{A Report on VAE GAN}
\hypersetup{
 pdfauthor={Gabriel Rolland},
 pdftitle={A Report on VAE GAN},
 pdfkeywords={},
 pdfsubject={},
 pdfcreator={Emacs 30.2 (Org mode 9.7.11)}, 
 pdflang={English}}
\begin{document}

\maketitle
\begin{abstract}
A report on the combination of Generative Adversarial Networks and
Variational Autoencoders.
This work is based on previous implementation of the VAE.
Based on (Larsen, Anders Boesen Lindbo and Sønderby, Søren Kaae and Larochelle, Hugo and Winther, Ole, 2016).
\end{abstract}
\section{GAN and VAE}
\label{sec:org1c7207e}
The point of the paper by Larsen et al. is to make a model
learn a feature representation instead of element-wise errors.
The question is how to learn this feature representation?
They do it by substituting the VAE decoder by a GAN.
Applying this to a dataset of faces, they show that the model
can extract higher order abstract features like "wearing glasses", "black hair", etc.
and apply them to generate new images with the requested features.
\section{Implementation}
\label{sec:org7eb2aae}
All the point is to collapse the decoder from our VAE into a GAN.
GAN are neural networks that play a minimax game where a generator (Gen) tries to fool a discriminator (Dis).
They show the following loss:
$$
\mathcal{L}_{GAN} = log(Dis(x)) +  log(1 - Dis(Gen(z)))
$$
That is, the loss consists of two terms:
\begin{enumerate}
\item How well the Discriminator plays \(log(Dis(x))\) and
\item How bad the Discriminator is fooled by the generator \(log(1 -Dis(Gen(z)))\)
\end{enumerate}
\section{References}
\label{sec:org743a79b}

\noindent
Larsen, Anders Boesen Lindbo and Sønderby, Søren Kaae and Larochelle, Hugo and Winther, Ole (2016). \emph{Autoencoding beyond pixels using a learned similarity metric}, arXiv.
\end{document}
